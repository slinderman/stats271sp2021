\documentclass[aspectratio=169]{beamer}
\usetheme{simple}

\RequirePackage[l2tabu, orthodox]{nag}


% \usepackage[left=1.in, right=1.in, top=1.25in, bottom=1.25in]{geometry}

% FONTS
%\usepackage[T1]{fontenc}

% Replace default Latin Modern typewriter with its proportional counterpart
% http://www.tug.dk/FontCatalogue/lmoderntypewriterprop/
%\renewcommand*\ttdefault{lmvtt}


%%% OPTION 1 - Fourier Math + New Century Schoolbook + ParaType Sans

% % Import Fourier Math (this imposes its own New Century Schoolbook type)
% % http://www.ctan.org/tex-archive/fonts/fouriernc/
%\usepackage{fouriernc}
%\usepackage{amsmath}
% % Replace with TeX Gyre Schola version of New Century Schoolbook (must scale!)
% % http://www.tug.dk/FontCatalogue/tgschola/
%\usepackage[scale=0.92]{tgschola}
%\usepackage[scaled=0.88]{PTSans}

%% OPTION 2 - MathDesign Math + Bitstream Charter + ParaType Sans

% Import MathDesign (this brings along Bitstream Charter)
% http://www.ctan.org/tex-archive/fonts/mathdesign/
\usepackage[bitstream-charter]{mathdesign}
\usepackage{amsmath}
\usepackage[scaled=0.92]{PTSans}


% %%% OPTION 3 - MTPRO 2 Math + Termes Times + ParaType Sans

% \usepackage{tgtermes}
% \usepackage{amsmath}
% \usepackage[subscriptcorrection,
%             amssymbols,
%             mtpbb,
%             mtpcal,
%             nofontinfo  % suppresses all warnings
%            ]{mtpro2}
% \usepackage{scalefnt,letltxmacro}
% \LetLtxMacro{\oldtextsc}{\textsc}
% \renewcommand{\textsc}[1]{\oldtextsc{\scalefont{1.10}#1}}
% \usepackage[scaled=0.92]{PTSans}

% Use default fonts here
\usepackage{amsmath}
\usepackage{amssymb}

% \usepackage{titling}

% % COLOR
% \usepackage[table,usenames,dvipsnames]{xcolor}
\definecolor{shadecolor}{gray}{0.9}

% SPACING and TEXT
\usepackage[final,expansion=alltext]{microtype}
\usepackage[english]{babel}
\usepackage[parfill]{parskip}
\usepackage{afterpage}
\usepackage{framed}
\usepackage{verbatim}
\usepackage{setspace}

\newenvironment{exercise}[1]
{
    \itshape
    \paragraph{Exercise: \textit{#1}}
}
{ 
}


% \usepackage[bottom]{footmisc}
\usepackage[symbol]{footmisc}
\renewcommand{\thefootnote}{\arabic{footnote}}


% FIGURES
\usepackage{graphicx}
\usepackage[labelfont={it, small}, 
            textfont={small,singlespacing},
            % justification={justified,RaggedRight},
            singlelinecheck=false,
            margin=0pt]{caption}
\usepackage[format=hang]{subcaption}
% \usepackage{ccaption}

% % APPENDIX FIGURES
% \usepackage{chngcntr}

% % TABLES
% \usepackage{booktabs}
% \usepackage{longtable}
% \usepackage{hhline}

% ALGORITHMS
\usepackage[algoruled]{algorithm2e}
\usepackage{listings}
\usepackage{fancyvrb}
\fvset{fontsize=\normalsize}

% % THEOREMS
\usepackage{amsthm}
\newtheorem{proposition}{Proposition}
% \newtheorem{lemma}{Lemma}

% % BIBLIOGRAPHY
\usepackage{natbib}

% HYPERREF
% \usepackage[colorlinks,linktoc=all]{hyperref}
% \usepackage[all]{hypcap}
% \hypersetup{citecolor=MidnightBlue}
% \hypersetup{linkcolor=black}
% \hypersetup{urlcolor=MidnightBlue}

% % CLEVEREF must come after HYPERREF
% \usepackage[nameinlink]{cleveref}

% % ACRONYMS
% \usepackage[acronym,smallcaps,nowarn]{glossaries}
% % \makeglossaries

% % COLOR DEFINITIONS
\newcommand{\red}[1]{\textcolor{BrickRed}{#1}}
\newcommand{\orange}[1]{\textcolor{BurntOrange}{#1}}
\newcommand{\green}[1]{\textcolor{OliveGreen}{#1}}
\newcommand{\blue}[1]{\textcolor{MidnightBlue}{#1}}
\newcommand{\gray}[1]{\textcolor{black!60}{#1}}

% LISTINGS DEFINTIONS
\lstdefinestyle{mystyle}{
    commentstyle=\color{OliveGreen},
    keywordstyle=\color{BurntOrange},
    numberstyle=\tiny\color{black!60},
    stringstyle=\color{MidnightBlue},
    basicstyle=\ttfamily,
    breakatwhitespace=false,
    breaklines=true,
    captionpos=b,
    keepspaces=true,
    numbers=left,
    numbersep=5pt,
    showspaces=false,
    showstringspaces=false,
    showtabs=false,
    tabsize=2
}
\lstset{style=mystyle}

\usepackage[colorinlistoftodos,
            prependcaption,
            textsize=small,
            backgroundcolor=yellow,
            linecolor=lightgray,
            bordercolor=lightgray]{todonotes}

\usepackage{soul}
\input{preamble/preamble_math.tex}
% \input{preamble/preamble_acronyms.tex}

\title{STATS271/371: Applied Bayesian Statistics}
\subtitle{Bayesian Mixture Models and (Collapsed) Gibbs Sampling}
\author{Scott Linderman}
\date{\today}


\begin{document}


\maketitle

\begin{frame}{Box's Loop}
\begin{center}
\includegraphics[width=.85\linewidth]{figures/lap1/boxsloop.jpeg}\\
\end{center} 
\begin{flushright}
{\footnotesize Blei, \textit{Ann. Rev. Stat. App.} 2014.}
\end{flushright}
\end{frame}

\begin{frame}{Lap 4: Bayesian Mixture Models and (Collapsed) Gibbs Sampling}
\begin{itemize}
    \item \hyperref[sec:mixtures]{\textbf{Model:} Bayesian mixture models}
    \item \hyperref[sec:gibbs]{\textbf{Algorithm:} Gibbs sampling}
    \item \hyperref[sec:ppcs]{\textbf{Criticism:} Posterior predictive checks}
    \item \hyperref[sec:collapsed_gibbs]{\textbf{Algorithm II:} Collapsed Gibbs sampling}
\end{itemize}
\end{frame}


\section{Model: Bayesian Mixture Models}
\label{sec:mixtures}

\begin{frame}{Motivation: Clustering scRNA-seq data}
\centering
\includegraphics[width=0.7\textwidth]{figures/lap4/scrnaseq.pdf}

From \citet{Kiselev2019-bt}
\end{frame}

\begin{frame}{Motivation: Foreground/background segmentation}
\centering
\includegraphics[width=0.9\textwidth]{figures/lap4/segmentation.png}

{\footnotesize \url{https://ai.stanford.edu/~syyeung/cvweb/tutorial3.html}}
\end{frame}

\begin{frame}{Motivation: Density estimation}
\centering
\includegraphics[width=0.65\textwidth]{figures/lap4/sphx_glr_plot_species_kde_001.png}
{\footnotesize \url{https://scikit-learn.org/stable/auto_examples/neighbors/plot_species_kde.html}}
\end{frame}

\begin{frame}{Notation}
    
\textbf{Constants: } Let
\begin{itemize}
    \item $N$ denote the number of data points.
    \item $K$ denote the number of mixture components (i.e. clusters)
\end{itemize}

\textbf{Data:} Let
\begin{itemize}
    \item $\mbx_n \in \reals^D$ denote the $n$-th data point.
\end{itemize}

\textbf{Latent Variables:} Let
\begin{itemize}
    \item $z_n \in \{1, \ldots, K\}$ denote the \textit{assignment} of the $n$-th data point.
\end{itemize}
\end{frame}

\begin{frame}{Notation II}

\textbf{Parameters:} Let
\begin{itemize}
    \item $\mbeta_k$ denote the \textit{natural parameters} of component $k$
    \item $\mbpi \in \Delta_K$ denote the component \textit{proportions} (i.e. probabilities).
\end{itemize}

\textbf{Hyperparameters:} Let
\begin{itemize}
    \item $\mbphi, \nu$ denote hyperparameters of the prior on $\mbeta$
    \item $\mbalpha \in \reals_+^{K}$ denote the concentration of the prior on proportions.
\end{itemize}
    
\end{frame}

\begin{frame}{Generative Model}

\begin{enumerate}
    \item Sample the proportions from a Dirichlet prior:
    \begin{align}
        \mbpi &\sim \distDirichlet(\mbalpha) \hspace{10em}
    \end{align}
    
    \item Sample the parameters for each component:
    \begin{align}
        \mbeta_k &\iid{\sim} p(\mbeta \mid \mbphi, \nu) \qquad \text{for } k = 1, \ldots, K
    \end{align}
    
    \item Sample the assignment of each data point:
    \begin{align}
        z_n &\iid{\sim} \mbpi \hspace{3em} \qquad \text{for } n = 1, \ldots, N
    \end{align}
    
    \item Sample data points given their assignments:
    \begin{align}
        \mbx_n &\sim p(\mbx \mid \mbeta_{z_n}) \qquad \text{for } n = 1, \ldots, N
    \end{align}
\end{enumerate}
    
\end{frame}

\begin{frame}{Joint distribution}
    This generative model corresponds to the following factorization of the joint distribution,
    \begin{align}
        p(\mbpi, \{\mbeta_k\}_{k=1}^K, \{(z_n, \mbx_n)\}_{n=1}^N \mid \mbphi, \nu, \mbalpha) 
        &= p(\mbpi \mid \mbalpha) \prod_{k=1}^K p(\mbeta_k \mid \mbphi, \nu) \, \prod_{n=1}^N p(z_n \mid \mbpi) \, p(\mbx_n \mid \mbz_n, \{\mbeta_k\}_{k=1}^K)
    \end{align}
    
    Equivalently, 
    \begin{multline}
        p(\mbpi, \{\mbeta_k\}_{k=1}^K, \{(z_n, \mbx_n)\}_{n=1}^N \mid \mbphi, \nu, \mbalpha) 
        = \\
        p(\mbpi \mid \mbalpha) \prod_{k=1}^K p(\mbeta_k \mid \mbphi, \nu) \, \prod_{n=1}^N \prod_{k=1}^K \left[ \Pr(z_n = k \mid \mbpi) \, p(\mbx_n \mid \mbeta_k) \right]^{\bbI[z_n = k]}
    \end{multline}
    
    Substituting in the assumed forms 
    \begin{align}
        p(\mbpi, \{\mbeta_k\}_{k=1}^K, \{(z_n, \mbx_n)\}_{n=1}^N \mid \mbphi, \mbalpha) 
        =
        \distDirichlet(\mbpi \mid \mbalpha) \prod_{k=1}^K p(\mbeta_k \mid \mbphi, \nu) \, \prod_{n=1}^N \prod_{k=1}^K \left[ \pi_k \, p(\mbx_n \mid \mbeta_k) \right]^{\bbI[z_n = k]}
    \end{align}
\end{frame}

\begin{frame}{Exponential family mixture models}
    What about $p(\mbx \mid \mbeta_k)$ and $p(\mbeta_k \mid \mbphi, \nu)$?
    
    Recall the \textit{exponential family} distributions from Lap 2. Let's assume an exponential family likelihood,
    \begin{align}
        p(\mbx \mid \mbeta_k) &= h(\mbx_n) \exp \left \{\langle t(\mbx_n), \mbeta_k \rangle - A(\mbeta_k) \right \}.
    \end{align}
    
    Then assume a \textit{conjugate prior},
    \begin{align}
        p(\mbeta_k \mid \mbphi, \nu) &\propto \exp \left \{ \langle \mbphi, \mbeta_k \rangle - \nu A(\mbeta_k) \right \}.
    \end{align}
    
    The hyperparmeters $\mbphi$ are \textit{pseudo-observations} of the sufficient statistics (like statistics from fake data points) and $\nu$ is a \textit{pseudo-count} (like the number of fake data points).
    
    Note that the product of prior and likelihood remains in the same family as the prior. That's why we call it conjugate.
\end{frame}

\begin{frame}{Example: Gaussian mixture model}

Assume the conditional distribution of $\mbx_n$ is a Gaussian with mean $\mbeta_{z_n} \in \reals^D$ and identity covariance,
\begin{align}
    p(\mbx_n \mid \mbeta_k) &= \cN(\mbx_n \mid \mbeta_{k}, \mbI) \\
    &= (2\pi)^{-D/2} \exp \left \{-\tfrac{1}{2} (\mbx_n - \mbeta_k)^\top (\mbx_n - \mbeta_k) \right\} \\
    &= (2\pi)^{-D/2} \exp \left \{-\tfrac{1}{2} \mbx_n^\top \mbx_n + \mbx_n^\top \mbeta_k - \tfrac{1}{2} \mbeta_k^\top \mbeta_k \right\},
\end{align}
which is an exponential family distribution with base measure $h(\mbx_n) = (2\pi)^{-D/2} e^{-\tfrac{1}{2}\mbx_n^\top \mbx_n}$, sufficient statistics $t(\mbx_n) = \mbx_n$, and log normalizer $A(\mbeta_k) = \tfrac{1}{2} \mbeta_k^\top \mbeta_k$.

Then assume a Gaussian prior on the component parameters. It's conjugate,
\begin{align}
    p(\mbeta_k \mid \mbphi, \nu) &= \cN(\nu^{-1} \mbphi, \nu^{-1} \mbI) 
    \propto \exp \left\{\mbphi^\top \mbeta_k -\tfrac{\nu}{2} \mbeta_k^\top \mbeta_k \right\}
    = \exp \left\{\mbphi^\top \mbeta_k -\nu A(\mbeta_k) \right\}.
\end{align}
Note that $\mbphi$ sets the location and $\nu$ sets the precision (i.e. inverse variance). 
    
\end{frame}

\section{Algorithm: MAP inference via coordinate ascent}
\label{sec:map_inference}

\begin{frame}{MAP inference via coordinate ascent}
Before diving into fully Bayesian inference algorithms, let's first consider \textbf{MAP inference}. 

\textbf{Idea: } find the mode of $p(\mbpi, \{\mbeta_k\}_{k=1}^K, \{z_n\}_{n=1}^N \mid \{\mbx_n\}_{n=1}^N, \mbphi, \nu, \mbalpha)$ by \textbf{coordinate ascent}.

For now, set $\mbphi= \mbzero$, and $\nu=0$ so that the prior is an (improper) uniform distribution. Then maximizing the posterior is equivalent to maximizing the likelihood. 

While we're simplifying, let's even fix $\mbpi = \tfrac{1}{K} \mbone_K$.

\end{frame}

\begin{frame}{Coordinate ascent in the Gaussian mixture model}
For the Gaussian mixture model (with uniform prior and $\mbpi = \tfrac{1}{K} \mbone_K$), coordinate ascent amounts to:
\begin{enumerate}
    \item For each $n=1,\ldots, N$, fix all variables but $z_n$ and find $z_n^\star$ that maximizes
    \begin{align}
        p(\mbpi, \{\mbeta_k\}_{k=1}^K, \{(z_n, \mbx_n)\}_{n=1}^N \mid \mbphi, \nu, \mbalpha)
        &\propto p(\mbx_n \mid z_n, \{\mbeta_k\}_{k=1}^K) 
        = \cN(\mbx_n \mid \mbeta_{z_n}, \mbI)
    \end{align}
    The cluster assignment that maximizes the likelihood is the one with the closest mean to $\mbx_n$, so~set
    \begin{align}
        z_n^\star &= \argmin_{k \in \{1,\ldots, K\}} \|\mbx_n - \mbeta_k\|_2.
    \end{align}
\end{enumerate}
\end{frame}

\begin{frame}{Coordinate ascent in the Gaussian mixture model II}
\begin{enumerate}
    \item[2] For each $k=1,\ldots,K$, fix all variables but $\mbeta_k$ and find $\mbeta_k^\star$ that maximizes,
    \begin{align}
        p(\mbpi, \{\mbeta_k\}_{k=1}^K, \{(z_n, \mbx_n)\}_{n=1}^N \mid \mbphi, \nu, \mbalpha)
        &\propto \prod_{n=1}^N p(\mbx_n \mid \mbeta_k)^{\bbI[z_n=k]} \\
        &\propto \exp \left\{ \sum_{n=1}^N \bbI[z_n=k] \left(\mbx_n^\top \mbeta_k - \tfrac{1}{2} \mbeta_k^\top \mbeta_k \right) \right\}
    \end{align}
    Taking the derivative of the log and setting to zero yields,
    \begin{align}
        \mbeta_k^\star &= \frac{1}{N_k} \sum_{n=1}^K \bbI[z_n=k] \mbx_n,
    \end{align}
    where $N_k = \sum_{n=1}^N \bbI[z_n=k]$.
\end{enumerate}

This is the \textbf{k-means algorithm}!

\end{frame}

\begin{frame}{Aside: EM in the Gaussian mixture model}

We'll talk more about \textit{coordinate ascent variational inference}~(CAVI) and \textit{expectation-maximization}~(EM) next week. Not to spoil the surprise, but we'll see that they have a similar flavor. Instead of assigning~$z_n^\star$ to the closest cluster, we compute \textit{responsibilities} for each cluster:
\begin{enumerate}
    \item For each data point $n$ and component $k$, set the \textit{responsibility} to,
    \begin{align}
        \omega_{nk} = \frac{\pi_k \cN(\mbx_n \mid \mbeta_k, \mbI)}{\sum_{j=1}^K \pi_j \cN(\mbx_n \mid \mbeta_j, \mbI)}.
    \end{align}
    
    \item For each component $k$, set the mean to
    \begin{align}
        \mbeta_k^\star &= \frac{1}{N_k} \sum_{n=1}^K \omega_{nk} \mbx_n,
    \end{align}
    where $N_k = \sum_{n=1}^N \omega_{nk}$.
\end{enumerate}

Note that EM allows for arbitrary proportions $\mbpi$. Those can be updated as well: for each component $k$, set $\pi_k = \frac{N_k}{N}$.

\end{frame}

\section{Algorithm: Gibbs Sampling}
\label{sec:gibbs}


\begin{frame}{Lap 4: Bayesian Mixture Models and (Collapsed) Gibbs Sampling}
\begin{itemize}
    \item \hyperref[sec:mixtures]{Model: Bayesian mixture models}
    \item \hyperref[sec:gibbs]{\textbf{Algorithm: Gibbs sampling}}
    \item \hyperref[sec:ppcs]{Criticism: Posterior predictive checks}
    \item \hyperref[sec:collapsed_gibbs]{Algorithm II: Collapsed Gibbs sampling}
\end{itemize}
\end{frame}

\begin{frame}[t]{Gibbs sampling in Bayesian mixture models}
\textbf{Idea: } just like in coordinate ascent, update one variable at a time. \textit{But rather than setting it to its conditional mode, sample from its conditional distribution.}
\end{frame} 
    
\begin{frame}{Gibbs sampling in Bayesian mixture models II}
\begin{enumerate}
    \item For each data point $n$, sample a new assignment from the complete conditional distribution
    \begin{align}
        z_n &\sim p(z_n \mid \{\mbx_n\}_{n=1}^N, \{z_{n'}\}_{n'\neq n}, \{\mbeta_k\}_{k=1}^K, \mbpi, \mbphi, \nu, \mbalpha).
    \end{align}
    Thanks to the factorization of the joint distribution,
    \begin{align}
        \Pr(z_n = k \mid -) 
        &\propto \Pr(z_n =k \mid \mbpi) \, p(x_n \mid \mbeta_k)
    \end{align}
    In the Gaussian mixture model, this is,
    \begin{align}
        \Pr(z_n = k \mid -) 
        &= \frac{\pi_k \cN(\mbx_n \mid \mbeta_k, \mbI)}{\sum_{j=1}^K \pi_j \cN(\mbx_n \mid \mbeta_j, \mbI)} \\
        &\equiv \omega_{nk}.
    \end{align}
    I.e., Gibbs sampling generates \textit{random} assignments by sampling according to the responsibilities. 
\end{enumerate}
\end{frame} 

\begin{frame}{Gibbs sampling in Bayesian mixture models III}
\begin{enumerate}
    \item[2] For each component $k$, sample new parameters from their complete conditional,
    \begin{align}
        \mbeta_k &\sim p(\mbeta_k \mid \{(\mbx_n, z_n)\}_{n=1}^N, \{\mbeta_{k'}\}_{k'\neq k}, \mbpi, \mbphi, \nu, \mbalpha).
    \end{align}
    Thanks to the factorization of the joint distribution,
    \begin{align}
        p(\mbeta_k \mid -) &\propto p(\mbeta_k \mid \mbphi, \nu) \prod_{n: z_n=k} p(x_n \mid \mbeta_k).
    \end{align}
    In an Gaussian mixture model,
    \begin{align}
        p(\mbeta_k \mid -) &\propto \exp \left\{ \Big( \mbphi + \sum_{n=1}^N \bbI[z_n=k] \mbx_n\Big)^\top \mbeta_k  - \frac{\nu + N_k}{2} \mbeta_k^\top \mbeta_k \right\} \\
        &\propto \cN\left(\mbeta_k \mid (\nu + N_k)^{-1} \Big(\mbphi + \sum_{n=1}^N \bbI[z_n=k] \mbx_n\Big), \, (\nu + N_k)^{-1} \mbI \right)
    \end{align}
    where $N_k = \sum_{n=1}^N \bbI[z_n =k]$. What happens when $N_k \to \infty$?
\end{enumerate}
\end{frame}

\begin{frame}{Gibbs sampling in Bayesian mixture models IV}
\begin{enumerate}
    \item[3] Finally, sample new component proportions from their complete conditional,
    \begin{align}
        \mbpi &\sim p(\mbpi \mid \{(\mbx_n, z_n)\}_{n=1}^N, \{\mbeta_{k}\}_{k=1}^K, \mbphi, \nu, \mbalpha).
    \end{align}
    Thanks to the factorization of the joint distribution,
    \begin{align}
        p(\mbpi \mid -) &\propto \distDirichlet(\mbpi \mid \mbalpha) \prod_{n=1}^N p(z_n \mid \mbpi) \\
        &\propto \prod_{k=1}^K \pi_k^{\alpha_k - 1} \times \prod_{n=1}^N \prod_{k=1}^K \pi_k^{\bbI[z_n = k]} \\
        &\propto \distDirichlet \left(\mbpi \mid [\alpha_1 + N_1, \ldots, \alpha_K + N_K] \right)
    \end{align}
    where $N_k = \sum_{n=1}^N \bbI[z_n =k]$. What happens when $N_k \to \infty$?
\end{enumerate}
\end{frame}

\begin{frame}{Gibbs sampling in Bayesian exponential family mixture models}
What happens in general exponential family models? Step 2 becomes,
\begin{enumerate}
    \item[2] For each component $k$, sample new parameters from their complete conditional,
    \begin{align}
        p(\mbeta_k \mid -) 
        &\propto \exp \left\{ \Big\langle \mbphi + \sum_{n=1}^N \bbI[z_n=k] t(\mbx_n), \mbeta_k \Big \rangle - (\nu + N_k) A(\mbeta_k) \right\} \\
        &= p \left(\mbeta_k \,\Big|\, \mbphi + \sum_{n=1}^N \bbI[z_n=k] \, t(\mbx_n), \nu + N_k \right)
    \end{align}
    where $N_k = \sum_{n=1}^N \bbI[z_n =k]$. 
\end{enumerate}
\end{frame}

\begin{frame}{Opportunities for parallelism}
    
\begin{itemize}
    \item In all three algorithms above, the updates of $z_n$ are independent of one another (once you fix $\{\mbeta_k\}_{k=1}^K$) and hence can be performed in parallel.
    
    \item Likewise, the updates of $\mbeta_k$ are independent of one another (once you fix $\{z_n\}_{n=1}^N$) and hence can be performed in parallel.
    
    \item In fact, we can write these as simple map-reduce algorithms and take advantage of parallel hardware if it's available.
    
    \item In the Gibbs sampling case, updating many variables at once from their combined conditional distribution is called \textbf{blocked Gibbs sampling}, and it's particularly easy when the variables are conditionally independent, as in the Bayesian mixture model.
\end{itemize}
\end{frame}

\begin{frame}{Next time}

\begin{itemize}
    \item Show that \textbf{Gibbs is a special case of MH} and, as such, asymptotically generates samples from the posterior distribution.
    
    \item Talk about \textbf{posterior predictive checks} and ways of choosing $K$.
    
    \item Introduce \textbf{collapsed Gibbs sampling}, which will allow us to generalize to \textbf{nonparametric Bayesian mixture models}.
\end{itemize}
    
\end{frame}

\section{Criticism: Posterior predictive checks}
\label{sec:ppcs}

\section{Algorithm II: Collapsed Gibbs Sampling}
\label{sec:collapsed_gibbs}

\begin{frame}[t,allowframebreaks]
        \frametitle{References}
        \bibliographystyle{unsrtnat}
        \bibliography{refs.bib}
\end{frame}

\end{document}